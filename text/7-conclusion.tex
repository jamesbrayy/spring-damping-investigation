\section{Conclusion}
\setlength{\parindent}{15pt}

This investigation aimed to experimentally determine how spring configuration and attached mass influence the rate of damping, quantified by the exponential decay constant $\gamma$, in a vertically oscillating mass-spring system. The experiment involved the manipulation of the load mass attached to a spring system and measurement of displacement--over--time for the mass, leading to the determination of the coefficients $\gamma$ and $\omega'$.

 My hypothesis was that if the attached mass $m{\textsubscript{load}}$ is increased, there will be a decrease in the decay constant $\gamma$ proportional to $m{\textsubscript{load}}^{-1}$, and if the spring system is configured in series, there will be no change in the decay constant. The results confirmed the inverse relationship between the decay constant and the load mass, with larger masses exhibiting slower exponential decay, consistent with the hypothesised proportionality.

To check the precision of the experiment, the spring constant of each spring configuration was determined, demonstrating that there was very little relative error between trials.

For different spring configurations, the calculated values of $b$ from the graph of $\gamma$ and $m{\textsubscript{load}}^{-1}$ showed small but statistically significant variation at the 5\% level between series and single-spring setups, indicating that the spring arrangement had a tangible influence on decay constant under the conditions tested. This could be due to slight differences in friction or internal damping within the springs themselves, or small misalignments in the experimental setup that altered the effective damping. While there is a degree of uncertainty and a limited number of datapoints, the experiment stands as relatively sound, with sources of errors acknowledged and improvements suggested.

Overall, the results confirm that the decay constant $\gamma$ is inversely related to the effective mass and that spring configuration has an indirect yet measurable effect on damping, providing empirical support for the theoretical model.