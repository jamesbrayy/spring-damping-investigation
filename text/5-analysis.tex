\section{Analysis of Results}

Analysis of the data in presented by \textbf{Graph 1} finds reasonably clear patterns emerging. By plotting the sum of the damped angular frequency squared and decay constant squared ($\omega'^2 + \gamma^2$) against $m_{\text{load}}^{-1}$, a highly linear relationship between the variables can be observed. Forcing a linear line of best fit, the gradient is expected to yield the spring constant $k$. From the recorded data, the calculated spring constants are:

\vspace{-1em}

\begin{align*}
k_{\text{light}} &= 23.1 \si{\kilo\gram\per\second\squared} \\
k_{\text{single stiff}} &= 53.1 \si{\kilo\gram\per\second\squared} \\
k_{\text{series stiff}} &= 26.5 \si{\kilo\gram\per\second\squared}
\end{align*}

Given spring constant reflects the force required to stretch the spring a certain distance, it follows that the stiff spring has a higher $k$ value than the light spring. Furthermore, the precision of the data can be testified by application of the general spring constant equation that states that the $k_{\text{equivalent}}$ of two identical springs in series is equal to $\tfrac{1}{2}k_{\text{single stiff}}$. From the measured $k_{\text{single stiff}}$, the theoretical equivalent series spring constant is $\tfrac{1}{2} \left( 53.1 \right) = 26.6 \; \text{(3 s.f.)}$. This possesses a highly respectable percentage error of $- 0.376 \%$ between the calculated and empirical values, solidifying the relative precision between calculations.

\textbf{Graph 2} depicts a slightly more ambiguous pattern. Unlike the clear proportionality observed in \textbf{Graph 1}, the relationship between $\gamma$ and $m_{\text{load}}^{-1}$ shows greater scatter, and the physical interpretation of the linear fit is less direct. Both the stiff spring and series stiff spring datasets still yield strong linear correlations ($R^2 = 0.9942$ and $0.9891$ respectively), but the fitted lines have small but non-negligible intercepts, one positive and one negative. These offsets suggest random influences outside the idealised damping model, such as frictional or measurement effects that contribute to a baseline damping constant independent of mass.

\begin{align*}
\intertext{Let $m_{\mathrm{eq}} = m_{\mathrm{load}} + \tfrac{1}{3} m_{\mathrm{spring}}$ and $\lambda = m_{\mathrm{load}}^{-1}$}
\gamma &= \frac{b}{2 m_{\mathrm{eq}}} \\
       &= \frac{b}{2 \left(m_{\mathrm{load}} + \tfrac{1}{3} m_{\mathrm{spring}}\right)} \\
       &= \frac{b}{2 \left(\tfrac{1}{\lambda} + \tfrac{1}{3} m_{\mathrm{spring}}\right)} \\
       &= \frac{b}{2} \frac{\lambda}{1 + \tfrac{1}{3} m_{\mathrm{spring}} \, \lambda} \\
\therefore \frac{d \gamma}{d \lambda} &= \frac{b}{2} \frac{1}{\left(1 + \tfrac{1}{3} m_{\mathrm{spring}} \, \lambda\right)^2}
\end{align*}

\begin{align*}
\intertext{$\bar{b}_{\textsubscript{single}} = 4.22 \times 10^{-3} \si{\kilo\gram\per\second}, s.e.\left(b_{\textsubscript{single}} \right) = 0.00018$}
\text{C}_{95\%}\left(b_{\textsubscript{single}} \right) &= \bar{b}_{\textsubscript{single}} \pm (1.96 \times {s.e.\left(b_{\textsubscript{single}} \right)}) \\
&= 4.22 \times 10^{-3} \pm (1.96 \times 0.00018) \\
&= \left( 4.22 \pm 0.353 \right) \times 10^{-3} \\
&= \left( 3.87, \; 4.57 \right) \times 10^{-3} \; \si{\kilo\gram\per\second}
\intertext{$\bar{b}_{\textsubscript{series}} = 3.40 \times 10^{-3} \si{\kilo\gram\per\second}, s.e.\left(b_{\textsubscript{series}} \right) = 0.00020$}
\text{C}_{95\%}\left(b_{\textsubscript{series}} \right) &= \bar{b}_{\textsubscript{series}} \pm (1.96 \times {s.e.\left(b_{\textsubscript{series}} \right)}) \\
&= 3.40 \times 10^{-3} \pm (1.96 \times 0.00020) \\
&= \left( 3.40 \pm 0.392 \right) \times 10^{-3} \\
&= \left( 3.01, \; 3.79 \right) \times 10^{-3} \; \si{\kilo\gram\per\second} \\
\end{align*}

\vspace{-2em}

\captionof{figure}{Calculations of 95\% confidence intervals for the $b$ values of the single and series configurations.}

\vspace{1em}

\lipsum[10-13]
